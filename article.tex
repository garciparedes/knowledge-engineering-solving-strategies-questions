\documentclass[10pt, a4paper,spanish]{article}
\usepackage[utf8]{inputenc}

\usepackage{lipsum} % Package to generate dummy text throughout this template
\usepackage{varwidth}
\usepackage{hyperref}
\usepackage{graphicx}

\usepackage[T1]{fontenc} % Use 8-bit encoding that has 256 glyphs
\usepackage{microtype} % Slightly tweak font spacing for aesthetics

\usepackage[hmarginratio=1:1,top=32mm,columnsep=20pt]{geometry} % Document margins
\usepackage[hang, small,labelfont=bf,up,textfont=it,up]{caption} % Custom captions under/above floats in tables or figures
\usepackage{booktabs} % Horizontal rules in tables
\usepackage{float} % Required for tables and figures in the multi-column environment - they need to be placed in specific locations with the [H] (e.g. \begin{table}[H])
\usepackage{hyperref} % For hyperlinks in the PDF

\usepackage{lettrine} % The lettrine is the first enlarged letter at the beginning of the text
\usepackage{paralist} % Used for the compactitem environment which makes bullet points with less space between them

\usepackage{abstract} % Allows abstract customization
\renewcommand{\abstractnamefont}{\normalfont\bfseries} % Set the "Abstract" text to bold
\renewcommand{\abstracttextfont}{\normalfont\small\itshape} % Set the abstract itself to small italic text

\usepackage{titlesec} % Allows customization of titles
\renewcommand\thesection{\Roman{section}} % Roman numerals for the sections
\renewcommand\thesubsection{\Roman{subsection}} % Roman numerals for subsections
\titleformat{\section}[block]{\large\scshape\centering}{\thesection.}{1em}{} % Change the look of the section titles
\titleformat{\subsection}[block]{\large}{\thesubsection.}{1em}{} % Change the look of the section titles
\usepackage{enumitem}

\usepackage{fancyhdr} % Headers and footers
\pagestyle{fancy} % All pages have headers and footers
\fancyhead{} % Blank out the default header
\fancyfoot{} % Blank out the default footer
\fancyhead[C]{ \today \ $\bullet$ ICO $\bullet$ Tutoría 1: Estrategias de resolución} % Custom header text
\fancyfoot[RO,LE]{\thepage} % Custom footer text

%----------------------------------------------------------------------------------------
%	TITLE SECTION
%----------------------------------------------------------------------------------------

\title{\vspace{-15mm}\fontsize{24pt}{10pt}\selectfont\textbf{Tutoría 1: Estrategias de resolución}} % Article title

\author{Sergio García Prado}
\date{\today}

%----------------------------------------------------------------------------------------

\begin{document}

	\maketitle % Insert title

	\thispagestyle{fancy} % All pages have headers and footers


%----------------------------------------------------------------------------------------
%	TEXT
%----------------------------------------------------------------------------------------

	\section{Cuestiones}

		\subsection{A qué tipo de búsqueda da lugar la estrategia de resolución lineal:}

			\begin{enumerate}[label=\Alph*)]
				\item \textbf{Primero en profundidad}
				\item Primero en anchura
				\item Ninguna de las anteriores
			\end{enumerate}

			\paragraph{}
			La estrategia de resolución lineal da lugar a una búsqueda primero en profundidad entre las cláusulas que forman la Base de Conocimiento. La causa está originada por la restricción que implica que una resolución debe estar formada por una cláusula cualquiera y la última resolvente obtenida. Esto hace que se genere una búsqueda primero en profundidad hacia la clausula vacía.


		\subsection{?`Es necesaria la presencia de una cláusula unitaria en un conjunto de cláusulas para que exista una refutación por entrada?}

			\paragraph{}
			Si que lo es. La demostración se puede llevar a cabo debido a la equivalencia con la estrategia de resolución unitaria. Esta estrategia requiere que una de las cláusulas utilizadas en cada resolución sea unitaria, por lo tanto obliga a que al menos una de las del conjunto inicial sea unitaria.


	\section{Problemas}

		\subsection{Sea S el conjunto de cláusulas $ \{ P(x), \lnot P(A) \lor Q(A), P(x) \lor \lnot Q(x), \lnot P(x) \lor \lnot Q(x) \} $. ¿Es inconsistente el conjunto de cláusulas S? ?`Por qué?}

			\paragraph{}


		\subsection{Sea S el conjunto de cláusulas $ \{ P(B), \lnot P(A) \lor Q(A), P(x) \lor \lnot Q(x), \lnot P(x) \lor \lnot Q(x) \} $. ¿Es inconsistente el conjunto de cláusulas S? ?`Por qué?}

			\paragraph{}


\end{document}
