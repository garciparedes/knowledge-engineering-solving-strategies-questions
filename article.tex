\documentclass[10pt, a4paper,spanish]{article}
\usepackage[utf8]{inputenc}

\usepackage{lipsum} % Package to generate dummy text throughout this template
\usepackage{varwidth}
\usepackage{hyperref}
\usepackage{graphicx}

\usepackage[T1]{fontenc} % Use 8-bit encoding that has 256 glyphs
\usepackage{microtype} % Slightly tweak font spacing for aesthetics

\usepackage[hmarginratio=1:1,top=32mm,columnsep=20pt]{geometry} % Document margins
\usepackage[hang, small,labelfont=bf,up,textfont=it,up]{caption} % Custom captions under/above floats in tables or figures
\usepackage{booktabs} % Horizontal rules in tables
\usepackage{float} % Required for tables and figures in the multi-column environment - they need to be placed in specific locations with the [H] (e.g. \begin{table}[H])
\usepackage{hyperref} % For hyperlinks in the PDF

\usepackage{lettrine} % The lettrine is the first enlarged letter at the beginning of the text
\usepackage{paralist} % Used for the compactitem environment which makes bullet points with less space between them

\usepackage{abstract} % Allows abstract customization
\renewcommand{\abstractnamefont}{\normalfont\bfseries} % Set the "Abstract" text to bold
\renewcommand{\abstracttextfont}{\normalfont\small\itshape} % Set the abstract itself to small italic text

\usepackage{titlesec} % Allows customization of titles
\renewcommand\thesection{\Roman{section}} % Roman numerals for the sections
\renewcommand\thesubsection{\Roman{subsection}} % Roman numerals for subsections
\titleformat{\section}[block]{\large\scshape\centering}{\thesection.}{1em}{} % Change the look of the section titles
\titleformat{\subsection}[block]{\large}{\thesubsection.}{1em}{} % Change the look of the section titles
\usepackage{enumitem}

\usepackage{fancyhdr} % Headers and footers
\pagestyle{fancy} % All pages have headers and footers
\fancyhead{} % Blank out the default header
\fancyfoot{} % Blank out the default footer
\fancyhead[C]{ \today \ $\bullet$ ICO $\bullet$ Tutoría 1: Estrategias de resolución} % Custom header text
\fancyfoot[RO,LE]{\thepage} % Custom footer text

%----------------------------------------------------------------------------------------
%	TITLE SECTION
%----------------------------------------------------------------------------------------

\title{\vspace{-15mm}\fontsize{24pt}{10pt}\selectfont\textbf{Tutoría 1: Estrategias de resolución}} % Article title

\author{Sergio García Prado}
\date{\today}

%----------------------------------------------------------------------------------------

\begin{document}

	\maketitle % Insert title

	\thispagestyle{fancy} % All pages have headers and footers


%----------------------------------------------------------------------------------------
%	TEXT
%----------------------------------------------------------------------------------------

	\section{Cuestiones}


		\subsection{Definir la subsunción en lógica de primer orden e indicar cuándo se aplica la estrategia de eliminación de cláusulas subsumidas.}

			\paragraph{}
			Se denomina ligadura al par ordenado termino/variable y se interpreta diciendo que el término substituye las ocurrencias de la variable.

			\paragraph{}
			Se denomina substitución, s, a un conjunto finito de ligaduras con la restricción de que una variable no puede aparecer más de una vez ni en una ligadura ni en el conjunto de todas ellas.

			\paragraph{}
			Se dice que una clausula $k_1$  subsume a otra clausuala $k_2$ si y sólo si existe una substitución $s/k_1$ tal que s está contenido en $k_2$

			\paragraph{}
			La estrategia de eliminación de clausulas subsumidas es un método de simplificación que se puede aplicar en cada paso del proceso de resolución para así hacerlo mucho más eficiente.


		\subsection{?`Cuál es la intuición en la que se apoya la estrategia del conjunto soporte?}

			\paragraph{}
			La intuición en la que se apoya la estrategia del conjunto soporte es la drástica reducción del espacio de búsqueda aplicando la restricción de que cada resolución sea una combinación de una cláusula del conjunto soporte y con otra sentencia y añadiendo la resolvente a dicho conjunto soporte. Si se seleción bien el conjunto soporte inicial esta estrategia es completa. Un enfoque habitual es seleccionar la clausula del teorema como conjunto soporte inicial.


		\subsection{A qué tipo de búsqueda da lugar la estrategia de resolución lineal:}

			\begin{enumerate}[label=\Alph*)]
				\item \textbf{Primero en profundidad}
				\item Primero en anchura
				\item Ninguna de las anteriores
			\end{enumerate}

			\paragraph{}
			La estrategia de resolución lineal da lugar a una búsqueda primero en profundidad entre la Base de Conocimiento. La causa está originada por la restricción que implica que una resolución debe estar formada por una cláusula cualquiera y última resolvente. Esto hace que se genere una búsqueda primero en profundidad hacia la clausula vacía.


		\subsection{Indicar si la resolución lineal es completa utilizada con un procedimiento de extracción de respuesta sujeto a la restricción de que la pregunta es una conjunción de literales y todas las variables están cuantificadas existencialmente, con todos los cuantificadores al comienzo de la fórmula.}
			\paragraph{}
			Si que lo es dado que al negar el teorema(pregunta) este se convierte en una disyunción de literales negados todos ellos de forma universal, por lo tanto esto se puede utilizar como clausula central inicial, lo que resulta en una estrategia completa.


		\subsection{?`Es necesaria la presencia de una cláusula unitaria en un conjunto de cláusulas para que exista una refutación por entrada?}

			\paragraph{}
			Si que lo es. La demostración se puede llevar a cabo debido a la equivalencia con la estrategia de resolución unitaria. Esta estrategia requiere que una de las cláusulas utilizadas en cada resolución sea unitaria, por lo tanto obliga a que al menos una de las del conjunto inicial sea unitaria.


	\section{Problemas}


\end{document}
